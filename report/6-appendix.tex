\appendix
\appendixpage
\addappheadtotoc

\chapter{Recovery Figures}
\label{appendix:bitmap-errors}

To visually demonstrate some limitations of this error correction scheme, the following figures show the use of Reed-Solomon codes to repair bitmap images.

Figure (a) has 12966 errors and can be repaired, yet figure (b) has only 3288 errors and cannot be repaired.
This is because the errors in figure (a) are one contiguous burst, whereas the errors in figure (b) are many small bursts spread across the image, damaging more blocks than in figure (a).

\begin{figure}[H]
    \centering
    \begin{subcaptionbox}{A recoverable bitmap image with one corrupt area.}
        {\includegraphics[width=0.3\textwidth]{face_2.png}}
    \end{subcaptionbox}
    \hfill
    \begin{subcaptionbox}{An unrecoverable bitmap image with many small corrupt areas.}
        {\includegraphics[width=0.3\textwidth]{face_3.png}}
    \end{subcaptionbox}
    \hfill
    \begin{subcaptionbox}{Figure (a), repaired.}
        {\includegraphics[width=0.3\textwidth]{face_2_repaired.png}}
    \end{subcaptionbox}
    
    \caption{Image source: \texttt{scipy.datasets.face} (derived from \url{https://pixnio.com/fauna-animals/raccoons/raccoon-procyon-lotor})}
\end{figure}

\clearpage

\section{Image Generation Script}

\tiny
\lstinputlisting[language=Python, showstringspaces=false]{generate_figures.py}

\clearpage
