\chapter{Introduction}

\section{Reed-Solomon Codes}

Reed-Solomon codes are a well-known class of error-correcting codes used in a wide range of applications, from data storage to radio communication.
They are based on polynomials over finite fields. \cite{theory-of-error-correcting-codes}

The code used for this project is an erasure code over the field \GF{64}, implemented using $O(n \log n)$ algorithms introduced in \cite{novel-poly}.

The basic working principle of this type of Reed-Solomon code is the interpretation of data as values of a polynomial evaluated at points ${\omega_0, \omega_1, \ldots, \omega_{k - 1}}$ in a finite field $\text{GF}(2^n)$.
As there is only one polynomial of degree $k - 1$ or smaller passing through $k$ points, any combination of at least $k$ of the original and redundant points uniquely determines the original polynomial.

Any amount of redundancy can be added, limited only by the field size. As the chosen field is \GF{64}, the limit is effectively infinite.

An erasure code is a type of error-correcting code which requires that the locations of corrupted data are known.
The code cannot be used to discover the locations of corrupted data by itself.
In this case, hashes stored in the metadata of the parity file are used to determine error locations.

Other Reed-Solomon codes do locate errors without requiring hashes, but they are not used in this project, as hashes are a simpler and more efficient solution.

One common code is $\text{RS}(255, 223)$, which is used in CDs and DVDs, and uses 8-bit symbols (in the field $\text{GF}(2^8)$).
The notation $\text{RS}(n, k)$ denotes a code with $n$ total symbols, with $k$ data symbols and $n - k$ parity symbols.
The code used in this project has no fixed $n$ or $k$, they are specified by the user.

The code used in this project is also systematic, meaning that the original data is included in the output.
Non-systematic codes do not include the original data, so the receiver must decode the code to obtain the original data, even if no corruption occurred.

\section{Finite Fields}

Finite fields, also known as Galois fields, are mathematical structures which define addition, multiplication, subtraction, and division over a finite set of elements \cite{finite-fields-2nd-ed}
(as opposed to the more commonly known infinite fields, such as the rationals, reals, and complex numbers).

A field must satisfy the following properties:

\begin{itemize}[nosep]
    \item Associativity of addition and multiplication: $(a + b) + c = a + (b + c)$ and $(a \cdot b) \cdot c = a \cdot (b \cdot c)$
    \item Commutativity of addition and multiplication: $a + b = b + a$ and $a \cdot b = b \cdot a$
    \item Additive and multiplicative identity elements: $a + 0 = a$ and $a \cdot 1 = a$
    \item Additive inverses: for every $a$, there exists $-a$ such that $a + (-a) = 0$
    \item Multiplicative inverses: for every $a \neq 0$, there exists $a^{-1}$ such that $a \cdot a^{-1} = 1$
    \item Distributivity of multiplication over addition: $a \cdot (b + c) = a \cdot b + a \cdot c$
\end{itemize}

The theorems of polynomial mathematics used in Reed-Solomon codes only hold in a field, however standard computer arithmetic does not form a field.
Typical arithmetic supported natively by CPUs is fixed-size binary arithmetic with overflow, which is equivalent to arithmetic modulo a power of 2.
Modular arithmetic only forms a field with a prime modulus, so it cannot be used directly for Reed-Solomon codes.

For example, the operation $x * 2$ is not invertible, as it is equivalent to a left shift, from which the most significant bit of $x$ cannot be recovered.

Fortunately, it is possible to construct a field based on fixed-size integers, such as 64-bit integers.

In a finite field $\text{GF}(p^m)$, where $p$ is a prime number and $m$ is a positive integer, the elements are polynomials of degree $m - 1$, with coefficients in $\text{GF}(p)$.
For the \GF{n} case, an element in the field is interpreted as a polynomial with $n$ coefficients, where each coefficient is a bit (i.e. a value in $\text{GF}(2)$ = $\{0, 1\}$).
For the purposes of this project, $n$ is always 64, so the field is \GF{64}.

The notation $\omega_i$ is used to denote the integer $i$ converted to an element of the field \GF{64} by interpreting its bits as a polynomial, which is a no-op in code, as elements of \GF{64} are stored as 64-bit integers.

It is important to note that these polynomials are not the same as the ones used in Reed-Solomon codes to represent data and parity information.
Elements of \GF{n} are simply $n$ bit integers with more complex arithmetic. They are polynomials over $\text{GF}(2)$, with $n$ coefficients.
Reed-Solomon polynomials can be arbitrarily long. They are polynomials over \GF{64}, with an arbitrary number of coefficients, and each coefficient is itself a polynomial over \GF{2} with $64$ coefficients.

Finite field addition is defined as polynomial addition.
In a general field $\text{GF}(p^m)$, this would be implemented as pairwise addition of the coefficients of two polynomials, modulo $p$.

In binary finite fields, addition is equivalent to XOR, as the coefficients are bits. Therefore, $x + x = 0$, and $x = -x$.

Multiplication is defined as polynomial multiplication, followed by reduction modulo an irreducible polynomial of degree 64 (with $65$ coefficients, where the highest coefficient is $1$).

The irreductible polynomial used for this project is $x^{64} + x^4 + x^3 + x + 1$ \cite{low-weight-polynomials}.
The choice of irreducible polynomial does not affect correctness, and all possible fields corresponding to irreducible polynomials of degree 64 are isomorphic.

A simple but inefficient way to compute the multiplicative inverse of $x$ is to compute $x^{2^{64} - 2}$ using exponentiation by squaring.
In the early stages of this project, this was how the multiplicative inverse was calculated.

The extended Euclidean algorithm is more efficient, and is used in the final implementation.
