\chapter{Polynomial Oversampling and Recovery}

Standard algorithms for polynomial interpolation and evaluation, such as Newton interpolation and Horner's method, require $O(n^2)$ time.
Efficient $O(n \log n)$ algorithms are used instead, based on FFT-like transforms introduced in \cite{novel-poly}.

\section{Polynomial Basis}

The polynomial basis $\mathbb{X} = \{X_0, \ldots, X_{2^{64} - 1}\}$ admits transforms $\Psi_h^l$ and $(\Psi_h^l)^{-1}$ which convert between values at $h$ contiguous points with an arbitrary offset $l$ and coefficients in $\mathbb{X}$, with $h$ a power of two.

To encode a $RS(n, k)$ code, the data polynomial coefficients are obtained by applying $(\Psi_h^0)^{-1}$ to the input values, then additional values are obtained using $\Psi_h^l$ $\frac{n}{k}$ times at offsets $l = k, 2k, \ldots, n - k$.

The basis polynomials $X_i$ are defined as the products of polynomials $\hat{W}_j$ corresponding to the bits of the index $i$:
\[X_i = \prod_{j \in \text{bits}(i)} \hat{W}_j\]

$\hat{W}_i = \frac{W_i}{W_i(2^{i})}$ is a normalized vanishing polynomial of degree $2^{i}$, which vanishes (i.e. evaluates to zero) at the points $\omega_0, \omega_1, \ldots, \omega_{2^{i} - 1}$, and evaluates to $1$ at $\omega_{2^{i}}$.
\[\hat{W}_i(x) = \frac{W_i(x)}{W_i(2^{i})} = \frac{\prod_{j = 0}^{2^i - 1} (x - \omega_j)}{\prod_{j = 0}^{2^i - 1} (\omega_{2^i} - \omega_j)}\]

$\hat{W}_i$ has degree $2^{i}$, as it is the product of $2^{i}$ degree one factors divided by a constant. Therefore, $X_i$ has degree $i$, since it the product of $W_j$ corresponding to the bits set in $i$.
Since $\mathbb{X}$ contains $2^{64}$ polynomials with all degrees from $0$ to $2^{64} - 1$, it automatically is a valid basis for representing polynomials of degree up to $2^{64} - 1$.

All $W_i$ are linearized polynomials, which means they only have non-zero coefficients at powers-of-two indices and are additive:
\[W_i(x + y) = W_i(x) + W_i(y)\]

Note that the standard monomial basis $\{1, x, x^2, \ldots, x^{2^{64} - 1}\}$ could also be defined in a similar way, with $\hat{W}_i = X^{2^{i}}$, but that would not allow $O(n \log n)$ FFT-like transforms.

\section{Forward and Inverse Transforms}

Let $D_h$ be the data polynomial with $h$ coefficients $d_0, d_1, \ldots, d_{h - 1}$. It can be expressed as a recursive function $\Delta_i^m(x)$, with $D_h(x) = \Delta_0^0(x)$:
\[
\Delta_i^m(x) = \begin{cases}
    \Delta_{i+1}^m(x) + \hat{W}_i(x) \Delta_{i+1}^{m+2^i}(x) & 0 \leq i \le \log_2(h) \\
    d_m & i = \log_2(h) \\
    \end{cases}
\]

At each step, the polynomial is split into coefficients whose index has the $i$-th bit set and those which don't. The final steps select the coefficient corresponding to the selected index $m$.

%Because of the properties of $\hat{W}_i$, the vector of evaluations of $\Delta_0^0$ can be computed from two vectors of size $\frac{h}{2}$: the evaluations of $\Delta_1^0$ and $\Delta_1^1$ at even points
Because of the properties of the basis polynomials, the vector of $\frac{h}{2^i}$ evaluations of $\Delta_i^m$ can be computed from two vectors of size $\frac{h}{2^{i + 1}}$: the evaluations of $\Delta_{i+1}^m$ and $\Delta_{i+1}^{m + 2^i}$ at points with the $i + 1$ least significant bits unset.

Let $\Phi(i, m, l) = [\Delta_i^m(\omega_c + \omega_l) \text{ for } c \text{ in } [0, 2^i, \ldots, h - 2^i]]$ be the vector of $\frac{h}{2^i}$ evaluations of $\Delta_i^m$ at all points $\omega_c + \omega_l$ where $c$ has the $i$ most significant bits unset, with $l$ an arbitrary offset.

$\Phi(i, m, l)$ can be computed in $O(n)$ time from $\Phi(i + 1, m, l)$ and $\Phi(i + 1, m + 2^i, l)$.

For each pair of values at index $x$ in the two smaller vectors, the values at indices $2x$ and $2x + 2^i$ in the larger vector can be computed. The values will be denoted as $a, b, a', b'$ for clarity.

$a'$ is straightforwardly computed as:
\[a' = \Delta_i^m(\omega_c + \omega_l) = \Delta_{i+1}^m(\omega_c + \omega_l) + \hat{W}_i(\omega_c + \omega_l) \Delta_{i+1}^{m + 2^i}(\omega_c + \omega_l) = a + \hat{W}_i(\omega_c + \omega_l) b\]

The calculation of $b'$ relies on the properties of the vanishing polynomials:
\[b' = \Delta_i^{m}(\omega_c + \omega_l + \omega_{2^i}) = \Delta_{i+1}^m(\omega_c + \omega_l + \omega_{2^i}) + \hat{W}_i(\omega_c + \omega_l + \omega_{2^i}) \Delta_{i+1}^{m + 2^i}(\omega_c + \omega_l + \omega_{2^i})\]

The term $\omega_{2^i}$ vanishes in both $\Delta_{i+1}^m$ and $\Delta_{i+1}^{m + 2^i}$, since both contain only vanishing polynomials $W_j$ with $j \geq i + 1$.

As $\hat{W}_i$ is normalized, $\hat{W}_i(\omega_c + \omega_l + \omega_{2^i}) = \hat{W}_i(\omega_c + \omega_l) + \hat{W}_i(\omega_{2^i}) = \hat{W}_i(\omega_c + \omega_l) + 1$.

Therefore, $b'$ is computed as:
\[b' = a + (\hat{W}_i(\omega_c + \omega_l) + 1) b = a + \hat{W}_i(\omega_c + \omega_l) b + b = a' + b\]

The reverse calculation is also straightforward, and does not require division:
\[b' + a' = (a' + b) + a' = b\]
\[a' + \hat{W}_i(\omega_c + \omega_l) b = (a + \hat{W}_i(\omega_c + \omega_l) b) + \hat{W}_i(\omega_c + \omega_l) b = a\]

The vectors can be stored interleaved in a single array, initialized to $[d_0, d_1, \ldots, d_{h - 1}]$ ($h$ single-element vectors), and then updated in-place in $log_2(h)$ steps, each step requiring $O(n)$ time.

See the butterfly diagram in \cite{novel-poly} for a visual representation of the transforms.

In total, $n - 1$ unique factors are needed - one evaluation of $\hat{W}_{\log_2(n)}$, two of $\hat{W}_{\log_2(n) - 1}$, \ldots, $\frac{n}{2}$ evaluations of $\hat{W}_0$ - which can be computed in $O(n \log n)$ time.

The inverse and forward transforms are almost identical, except the outer loop direction and the inner operations are reversed.

\enlargethispage{\baselineskip} % squeeze the next line into the same page as the previous, since the transforms will completely fill the next page

Notice the transforms can use factors of a greater power than needed. To compute multiple transforms of different sizes with the same offset, only the factors for the largest size must be computed, and can be used for all smaller sizes.

\begin{algorithm}
    \caption{Transform Algorithms}
    \begin{algorithmic}
        \Function{PrecomputeFactors}{\text{pow}, \text{offset}}
            \State $\text{factors} \gets \text{new array of \GF{64} values of size } 2^{\text{pow}} - 1$
            \State $\text{factor\_idx} \gets 0$
            \For{$\text{step} \gets 0 \text{ to } \text{pow} - 1$}
                \State $\text{groups} \gets 2^{\text{pow} - \text{step} - 1}$
                \For{$\text{group} \gets 0 \text{ to } \text{groups} - 1$}
                    \State $\text{factors}[\text{factor\_idx}] \gets \hat{W}_{\text{step}}(\omega_{\text{group} \cdot 2^{\text{step} + 1}} + \omega_{\text{offset}})$
                    \State $\text{factor\_idx} \gets \text{factor\_idx} + 1$
                \EndFor
            \EndFor
            \State \Return $\text{factors}$
        \EndFunction
    \end{algorithmic}

    \begin{algorithmic}
        \Function{InverseTransform}{\text{data}, \text{factors}}
            \For{$\text{step} \gets 0 \text{ to } \log_2(\text{len(data)}) - 1$}
                \State $\text{group\_len} \gets 2^{\text{step}}$
                \State $\text{group\_factors\_start} \gets \text{len(factors)} + 1 - \frac{\text{len(factors)} + 1}{2^{\text{step}}}$
                \For{$\text{group} \gets 0 \text{ to } \frac{\text{len(data)}}{2^{\text{step} + 1}} - 1$}
                    \For{$\text{x} \gets 0 \text{ to } \text{group\_len} - 1$}
                        \State $i \gets \text{group} \cdot \text{group\_len} \cdot 2 + x$
                        \State $j \gets i + \text{group\_len}$
                        \State $\text{data}[j] \gets \text{data}[j] + \text{data}[i]$
                        \State $\text{data}[i] \gets \text{data}[i] + \text{data}[j] \cdot \text{factors}[\text{group\_factors\_start} + \text{group}]$
                    \EndFor
                \EndFor
            \EndFor
        \EndFunction
    \end{algorithmic}

    \begin{algorithmic}
        \Function{ForwardTransform}{\text{data}, \text{factors}}
            \For{$\text{step} \gets \log_2(\text{len(data)}) - 1 \text{ down to } 0$}
                \State $\text{group\_len} \gets 2^\text{step}$
                \State $\text{group\_factors\_start} \gets \text{len(factors)} + 1 - \frac{\text{len(factors)} + 1}{2^{\text{step}}}$
                \For{$\text{group} \gets 0 \text{ to } \frac{\text{len(data)}}{2^{\text{step} + 1}} - 1$}
                    \For{$\text{x} \gets 0 \text{ to } \text{group\_len} - 1$}
                        \State $i \gets \text{group} \cdot \text{group\_len} \cdot 2 + x$
                        \State $j \gets i + \text{group\_len}$
                        \State $\text{data}[i] \gets \text{data}[i] + \text{data}[j] \cdot \text{factors}[\text{group\_factors\_start} + \text{group}]$
                        \State $\text{data}[j] \gets \text{data}[j] + \text{data}[i]$
                    \EndFor
                \EndFor
            \EndFor
        \EndFunction
    \end{algorithmic}
\end{algorithm}

\section{Formal Derivative}

The error correction algorithm cannot directly use the inverse transform for interpolation, as the received data is not at contiguous points, and the number of non-corrupted points is likely not a power of two.

Instead, an algorithm based on the formal derivative is used which can recover the original polynomial from any $k$ intact points regardless of error location.

In all fields, the formal derivative of a polynomial is well-defined and the standard power and product rules apply, despite the concepts of limits and continuity not existing in finite fields.
\begin{gather*}
(f \cdot g)' = f' \cdot g + f \cdot g'\\
(\sum_{i = 0}^{n} a_i x^i)' = \sum_{i = 1}^{n} (i \cdot a_i) x^{i - 1}
\end{gather*}

The multiplication $i \cdot a_i$ between an integer and a field element is defined as repeated addition, which in \GF{n} is either zero or $a_i$, as $a_i + a_i = 0$.
\[
i \cdot a_i =
    \begin{cases}
        a_i & i\ \text{odd} \\
        0 & i\ \text{even}
    \end{cases}
\]

Therefore, the formal derivative of a polynomial $f$ in \GF{n} written in the standard monomial basis is:
\[f' = a_1 + a_3 x^2 + a_5 x^4 + \ldots\]

As the normalized vanishing polynomial $\hat{W}_i$ only has coefficients at power-of-two indices, the derivative will be a constant:
\[\hat{W}_i' = \frac{\prod_{j = 1}^{2^i - 1} \omega_j}{W_i(2^i)}\]

To find the derivative of the basis polynomial $X_i$, which is a product of up to 64 polynomials, the product rule generalized to a product of $n$ polynomials is used:

\[
(\prod_{i = 0}^{n} f_i)' = \sum_{j = 0}^{n} f_j' \cdot \prod_{i \neq j} f_j
\]

Therefore, the derivative of $X_i$ contains $|\text{bits}(i)|$ terms, each of which is a basis polynomial with one bit of $i$ unset, multiplied by the derivative of the vanishing polynomial corresponding to that bit:

\[
X_i' = \sum_{b \in \text{bits}(i)} \hat{W}_b' \cdot X_{i - 2^b}
\]

Notice that $X_i'$ only has terms with indices less than $i$, so the derivative of a polynomial in basis $\mathbb{X}$ can be computed in-place by iterating from the lowest degree to the highest degree coefficients, in $O(n \log n)$ time.

\begin{algorithm}
    \caption{Polynomial Derivative}
    \begin{algorithmic}
        \Function{PrecomputeDerivativeFactors}{\text{pow}}
            \State $\text{assert} 0 \leq \text{pow} \le 64$
            \State $\text{factors} \gets \text{new array of \GF{64} values of size } \text{pow}$
            \For{$l \gets 1 \text{ to } \text{pow} - 1$}
                \For{$j \gets 2^{l - 1} \text{ to } 2^l - 1$}
                    \State $\text{factors}[l] \gets \text{factors}[l] * \omega_j$
                \EndFor
                \If{$l + 1 \neq \text{pow}$}
                    \State $\text{factors}[l + 1] \gets \text{factors}[l]$
                \EndIf
                \State $\text{factors}[l] \gets \text{factors}[l] / W_l(2^l)$
            \EndFor
            \State \Return $\text{factors}$
        \EndFunction
    \end{algorithmic}
    \begin{algorithmic}
        \Function{FormalDerivative}{\text{data}, \text{factors}}
            \For{$i \gets 0 \text{ to } \text{len(data) - 1}$}
                \For{$\text{bit} \gets 0 \text{ to } \log_2(\text{len(data)})$}
                    %\State $\text{data}[i - 2^{\text{bit}}] \gets \text{data}[i] \cdot \text{factors}[\text{bit}]$
                    \If{$i \bitand 2^{\text{bit}} \neq 0$}
                        \State $\text{data}[i - 2^{\text{bit}}] \gets \text{data}[i - 2^{\text{bit}}] + \text{data}[i] \cdot \text{factors}[\text{bit}]$
                    \EndIf
                \EndFor
                \State $\text{data}[i] \gets 0$
            \EndFor
        \EndFunction
    \end{algorithmic}
\end{algorithm}

\section{Polynomial Recovery}

In order to recover the original polynomial using the formal derivative, an error locator polynomial is constructed which vanishes at the points where errors occurred.

Let $\text{ERASURES}$ be the set of indices where an error occurred. As previously mentioned, erasure codes require knowledge of the location of all errors, which, for this application, will be obtained using hashing.
\[e = \prod_{i \in \text{ERASURES}} (x + \omega_i)\]

Since $e$ does not depend on the actual values of the data polynomial, its values can be computed and multiplied with the received incomplete data polynomial $d$, since all unknown values will be zeroed.

The known values in $e \cdot d$ will be changed unpredictably by the multiplication, however the product rule allows the original polynomial $d$ to be recovered:
\begin{gather*}
(e \cdot d)' = e' \cdot d + e \cdot d'\\
(e \cdot d)'(\omega_x) = e'(\omega_x) \cdot d(\omega_x) + 0 \cdot d'(\omega_x)\ \forall\ x \in \text{ERASURES}\\
d(\omega_x) = \frac{(e \cdot d)'(\omega_x)}{e'(\omega_x)}\ \forall\ x \in \text{ERASURES}
\end{gather*}

Therefore, the original polynomial is recovered by multiplying the received polynomial by the error locator polynomial, applying the inverse transform, taking the formal derivative, applying the forward transform,
and finally dividing by the derivative of the error locator polynomial, at the error locations.

\begin{algorithm}
    \caption{Reed-Solomon Decoding}
    \begin{algorithmic}
        \State $(e, e') \gets \text{values of error locator polynomial and its derivative}$
        \State $\text{d} \gets [d_0, d_1, \ldots, d_{n - 1}]$ \Comment{received data}
        \State $\text{erasures} \gets [i_0, i_1, \ldots, i_k]$ \Comment{indices of errors}
        \State $\hat{d} \gets d \cdot e$ \Comment{multiply partially corrupt data by error locator polynomial}
        \State $\hat{d'} \gets \text{ForwardTransform}(\text{FormalDerivative}(\text{InverseTransform}(\hat{d})))$
        \For{$i \in \text{erasures}$}
            \State $d[i] \gets \hat{d'}[i] / e'[i]$
        \EndFor
    \end{algorithmic}
\end{algorithm}

\section{Error Locator Computation}

A $O(n \log n)$ algorithm for computing the error locator polynomial is described in \cite{novel-poly} which uses fast Walsh-Hadamard transforms, however it requires $2^r$ operations where $r$ is the power of the field, so it is not useful for \GF{64}.

Instead, I used a simpler $O(n \log^2 n)$ recursive algorithm which splits the polynomial into two halves, recursively computes the coefficients of each half, then multiplies the two halves in $O(n \log n)$ time using the inverse and forward transforms.

%TODO: Describe the algorithm in more detail
