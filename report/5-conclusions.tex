\chapter{Conclusions}

The implementation has been successfully tested and verified to generate parity data and repair errors in $O(n \log n)$ time (figure \ref{fig:benchmark_log_poly})
using Reed-Solomon codes based on finite field arithmetic and polynomial FFT-like transforms introduced in \cite{novel-poly}.

Correctness is verified through unit tests for the IO-free algorithms (\ref{appendix:unit}), as well as testing of the full encode-decode sequence on-disk with simulated file corruption.

The performance of the implementation is satisfactory, and benchmarks show that the process is mostly IO-bound (figure \ref{fig:end_to_end_benchmark}).

The fundamental arithmetic operations used are \GF{64} multiplication, which is made extremely fast by using carry-less multiplication,
with a speedup of over 50x compared to the Russian Peasant algorithm (table \ref{tab:arithmetic_benchmark}), and \GF{64} addition which is XOR.

\vspace{-1.5em}
\section{Future Improvement Directions}
\vspace{-0.5em}

The file metadata is not protected from corruption. This could be addressed by adding meta-parity blocks among the parity blocks, used to repair file header corruption, located using a special marker and hash placed in each meta-parity block.

The IO architecture could be extended to support using multiple files or entire directories, as input data, as well as to support multiple parity files - similar to PAR2 files - and single-file combined data and parity archives - similar to RAR archives.


