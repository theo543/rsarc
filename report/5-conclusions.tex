\chapter{Conclusions}

\section{Summary}

The file metadata is currently not protected from corruption.
This can be addressed by adding meta-parity blocks among the parity blocks.
Although the metadata is generally small, this is a significant flaw in the current implementation.

The implementation can successfully generate parity data and repair file corruption using Reed-Solomon codes in $O(n \log n)$ time.

Correctness is ensured using unit tests for the side-effect free part of the implementation - the finite field arithmetic and polynomial algorithms - and end-to-end testing for the entire encoding and decoding sequence, using simulated file corruption.

\pagebreak

\section{Figures}

% TODO: move this section to an annex

As a final note, and to visually demonstrate some limitations of this error correction scheme,
the following figures show the use of Reed-Solomon codes to repair bitmap images.

Figure B is impossible to repair, yet figure A has more errors.
This is because the errors in figure A are contiguous - they are burst errors - so they affect less blocks.

Figure C is the recovered image from figure A.

See the script \texttt{generate\_figures.py} for how these images were generated.

\begin{figure}[H]
    \centering
    \begin{subcaptionbox}{A bitmap image with repairable errors.}
        {\includegraphics[width=0.3\textwidth]{face_2.png}}
    \end{subcaptionbox}
    \hfill
    \begin{subcaptionbox}{A bitmap image with irreparable errors.}
        {\includegraphics[width=0.3\textwidth]{face_3.png}}
    \end{subcaptionbox}
    \hfill
    \begin{subcaptionbox}{Figure A, repaired.}
        {\includegraphics[width=0.3\textwidth]{face_2_repaired.png}}
    \end{subcaptionbox}
    
    \caption{Image source: \texttt{scipy.datasets.face}.}
\end{figure}
