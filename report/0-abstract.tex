\begin{abstractpage}

\begin{abstract}


The aim of this project is to implement file corruption detection and repair using Reed-Solomon error-correcting codes.

The code used is an erasure code over the field \GF{64}, implemented using $O(n \log n)$ transforms in a polynomial basis introduced by Sian-Jheng Lin, Wei-Ho Chung, and Yunghsiang S. Han in \cite{novel-poly}.

Finite field multiplication is implemented using carry-less multiplication, and division is implemented using the extended Euclidean algorithm.

The implementation is a command-line utility which can generate parity data for a file, and later detect and repair corruption in that file.

The file is split into $N$ blocks, and an arbitrary number of parity blocks $M$ can be generated, for a total of $N + M$ blocks.
Any $N$ blocks are sufficient to recover the original data, so up to $M$ corrupted blocks can be repaired.

As erasure codes require known error locations - the term 'erasure' refers to an error at a known location - a hash of each block is stored in the file header to detect corruption.

The file is not a single Reed-Solomon code, as that would require reading the entire file into memory, limiting the maximum file size which can be processed.
Instead, the file is interpreted as a matrix with $N + M$ rows, and each column is an separate Reed-Solomon code.

\end{abstract}

\end{abstractpage}

\begin{abstractpage}

\begin{rezumat}

\indent

Acest proiect are ca scop implementarea detecției și reparării coruperii fișierelor folosind coduri Reed-Solomon de corectare a erorilor.

Codul folosit este un cod de ștergere peste corpul \GF{64}, implementat folosind transformări în timp $O(n \log n)$ într-o bază polinomială introdusă de Sian-Jheng Lin, Wei-Ho Chung, și Yunghsiang S. Han în \cite{novel-poly}.

Înmulțirea în corp finit este implementată folosind înmulțire fără retenție ('carry-less'), iar împărțirea este implementată folosind algoritmul extins al lui Euclid.

Implementarea este un utilitar de linie de comandă care poate genera date de paritate pentru un fișier și, ulterior, să detecteze și repare corupție în fișier.

Fișierul este împărțit în $N$ blocuri, și un număr arbitrar de blocuri de paritate $M$ pot fi generate, pentru un total de $N + M$ blocuri.
Orice $N$ blocuri sunt suficiente pentru a recupera datele originale, deci cel mult $M$ blocuri corupte pot fi reparate.

Deoarece un cod de ștergere necesită cunoașterea locațiilor erorilor - termenul 'ștergere' înseamnă o eroare la o locație cunoscută - un hash al fiecărui bloc este stocat în antetul fișierului pentru a detecta corupție.

Fișierul nu este un singur cod Reed-Solomon, deoarece ar necesita citirea întregului fișier în memorie, limitând dimensiunea maximă de fișier care poate fi procesată.
De fapt, fișierul este interpretat ca o matrice cu $N + M$ rânduri, iar fiecare coloană este un cod separat Reed-Solomon.

\end{rezumat}

\end{abstractpage}
