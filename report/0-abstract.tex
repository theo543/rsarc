\begin{abstractpage}

\begin{abstract}


The aim of this project to implement file corruption detection and repair using Reed-Solomon error-correcting codes.

The code used is an erasure code over the field \GF{64}, implemented using \( O(n \log n) \) transforms in a polynomial basis introduced by Sian-Jheng Lin, Wei-Ho Chung, and Yunghsiang S. Han \cite{novel-poly}.

Finite field multiplication is implemented using carry-less multiplication, and division is implemented using the extended Euclidean algorithm.

The implementation is a CLI program which can generate parity data for a file, and later detect and repair corruption in that file.

The file is split into $N$ blocks, and an arbitrary number of parity blocks $M$ can be generated, for a total of $N + M$ blocks.
Any $N$ blocks are sufficient to recover the original data, so up to $M$ corrupted blocks can be repaired.

As erasure codes require known error locations, a hash of each block is stored in the file header to detect corruption.

The file is not a single Reed-Solomon code, as that would require reading the entire file into memory, limiting the maximum file size which can be processed.
Instead, the file is interpreted as a matrix with $N + M$ rows, and each column is an separate Reed-Solomon code.

\end{abstract}

\end{abstractpage}
