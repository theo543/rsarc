\chapter{Finite Field Arithmetic}

The constant $\text{POLYNOMIAL}$ referred to in pseudocode is the irreducible polynomial $x^{64} + x^4 + x^3 + x + 1$,
with the coefficient $x^{64}$ omitted, as it does not fit in a 64-bit integer. \cite{low-weight-polynomials}

As previously mentioned, Reed-Solomon codes require the use of non-standard arithmetic - arithmetic over finite fields.

In modular arithmetic, a modular inverse exists only for elements that are coprime with the modulus, i.e. only multiplication by odd numbers is invertible modulo $2^n$.
The field axioms require that all elements have a multiplicative inverse.

Using \GF{64} arithmetic is necessary to resolve this issue, but complicates the arithmetic.
Addition is extremely simple, as it is equivalent to XOR.
Multiplication, however, is less efficient than standard multiplication, and divison even less so.

\section{Russian Peasant Algorithm}

The Russian peasant algorithm multiplies two values in \GF{64} without requiring 128-bit integers.
It incrementally performs the multiplication by adding intermediate values into an accumulator, and slowly shifting the values to be multiplied and applying polynomial reduction.

The state of the algorithm consists of the two values to be multiplied $a$ and $b$, and an accumulator.

The algorithm must be executed at most 64 times, before $b$ is guaranteed to become zero. Then, the accumulator contains the result.

At each iteration, if the low bit of $b$ is set, the accumulator is XORed with $a$.
Then, $a$ is shifted left, and $b$ is shifted right.

This is justified because, at each step, we multiply the lowest coefficient of $b$ with $a$, and add the result (either $0$ or $a$) to the accumulator.
Then, moving on to the next coefficient of $b$, we divide $b$ by $x$ and multiply $a$ by $x$, which is equivalent to shifting $a$ left and $b$ right.

If the high bit of $a$ was set before shifting, $a$ is XORed with the irreducible polynomial.
This is because, conceptually, $a$ now has a 65th bit (a coefficient $x^{64}$), which requires reduction, done by subtracting the irreducible polynomial using XOR.

% pseudo code
\begin{algorithm}
\caption{Russian Peasant Multiplication}
\begin{algorithmic}
\Function{Multiply}{$a, b$}
\State $acc \gets 0$
% repeat 64 times
\For{$\text{i} \gets 1 \text{ to } 64$}
    \If{$b \bitand 1$}
        \State $acc \gets acc \oplus a$
    \EndIf
    \State $\text{carry} \gets a \bitand (1 \ll 63)$
    \State $a \gets a \ll 1$
    \State $b \gets b \gg 1$
    \If{$\text{carry}$}
        \State $a \gets a \oplus \text{POLYNOMIAL}$
    \EndIf
\EndFor
\State \Return $acc$
\EndFunction
\end{algorithmic}
\end{algorithm}


This algorithm is fairly simple and easy to implement, but multiplication can be done more efficiently on modern CPUs.
Still, this algorithm is necessary as a fallback, for CPUs which don't support 128-bit carry-less multiplication.

\section{Multiplication - Carry-less Multiplication}

\GF{64} multiplication can be performed using only three 128-bit carry-less multiplication operations.
Modern CPUs have support for this operation, as it is useful for cryptographic algorithms, computing checksums, and other applications. \cite{intel-clmul}

The terms "upper half" and "lower half" will be used to refer to the most significant 64 bits and least significant 64 bits of a 128-bit integer, respectively.

By multiplying $a$ and $b$ using carry-less multiplication, we obtain a 128-bit result.
We must reduce the upper half to a 64-bit result, which can then be XORed with the lower half to obtain the final result.

This can be done by multiplying the upper half of the result by the irreducible polynomial.
Then, the lower half of the result is the product reduced modulo the irreducible polynomial.

To understand why this works, consider the process of reduction.
The irreducible polynomial is aligned with each set bit in the upper half of the result, and XORed with the result.
This is effectively what carry-less multiplication does.

There is a complication, however.
A third multiplication is required to ensure full reduction, as the highest bits of the upper half can affect the lowest bits of the upper half.

For fields where $x^{n - 1} + 1$ is irreducible, two multiplications would suffice, but this is not the case for \GF{64}.

For example, consider $x^{127} + x^{67} + x^{66} + x^{64}$.
After aligning the irreducible polynomial with the highest bit and XORing, all bits in the upper half are zero.
At this point, the reduction is complete and should stop.
However, this is not how carry-less multiplication works.
The irreducible polynomial will also be aligned with the other three bits, and the lower half is XORed with some unnecessary values.

The upper half of the reduced result if and where this happened. A third multiplication will correct this.
The unnecessary XORs are undone by XORing with the lower half of the third multiplication.

The justification for the algorithm may seem somewhat complex, but the algorithm itself is very short, simple, and efficient.

The functions $\text{upper}(x)$ and $\text{lower}(x)$ return the upper and lower 64 bits of the 128-bit integer $x$, respectively.

\begin{algorithm}
\caption{Carry-less Multiplication}
\begin{algorithmic}
\Function{Multiply}{$a, b$}
\State $\text{result} \gets \text{CLMUL}(a, b)$
\State $\text{result\_partially\_reduced} \gets \text{CLMUL}(\text{upper}(\text{result}), \text{POLYNOMIAL})$
\State $\text{result\_fully\_reduced} \gets \text{CLMUL}(\text{upper}(\text{result\_partially\_reduced}), \text{POLYNOMIAL})$
\State \Return $\text{lower}(\text{result}) \oplus \text{lower}(\text{result\_partially\_reduced}) \oplus \text{lower}(\text{result\_fully\_reduced})$
\EndFunction
\end{algorithmic}
\end{algorithm}

\section{Extended Euclidean Algorithm}

The polynomial extended Euclidean algorithm, given polynomials $a$ and $b$, computes $s$ and $t$ such that $a \cdot s + b \cdot t = \text{gcd}(a, b)$.
When $b$ is set to the irreducible polynomial, $t$ is the multiplicative inverse of $a$. $s$ does not need to be computed.

The algorithm uses repeated Euclidean division.
Because the irreducible polynomial is of degree 64, the first Euclidean division iteration, in the first iteration of the Euclidean algorithm, is a special case.
As a 65-bit polynomial cannot fit in the 64-bit variable $b$, the first iteration is done manually, outside the loop.

In the following pseudocode, $\text{leading\_zeros}(x)$ returns the number of leading zero bits in $x$.
On modern CPUs, this function can be implemented using the \texttt{LZCNT} instruction.

\begin{algorithm}
\caption{Extended Euclidean Algorithm}
\begin{algorithmic}
\Function{ExtendedEuclidean}{$a$}

\State $\text{assert}(a \neq 0)$

\State \algorithmicif\ $a = 1$ \algorithmicthen\ \Return $1$ \algorithmicend \algorithmicif

\State $t \gets 0$
\State $\text{new\_t} \gets 1$
\State $r \gets \text{POLYNOMIAL}$
\State $\text{new\_r} \gets a$

\State $r \gets r \oplus (\text{new\_r} \ll (\text{leading\_zeros}(\text{new\_r}) + 1))$
\State $\text{quotient} \gets 1 \ll (\text{leading\_zeros}(\text{new\_r}) + 1)$

\While{$\text{new\_r} \neq 0$}
    \While{$\text{leading\_zeros}(\text{new\_r}) >= \text{leading\_zeros}(r)$}
        \State $\text{degree\_diff} \gets \text{leading\_zeros}(\text{new\_r}) - \text{leading\_zeros}(r)$
        \State $\text{r} \gets r \oplus (\text{new\_r} \ll \text{degree\_diff})$
        \State $\text{quotient} \gets \text{quotient} | (1 \ll \text{degree\_diff})$
    \EndWhile
    \State $(r, \text{new\_r}) \gets (\text{new\_r}, r)$
    \State $(t, \text{new\_t}) \gets (\text{new\_t}, t \oplus \text{gf64\_multiply}(\text{quotient}, \text{new\_t}))$
    \State $quotient \gets 0$
\EndWhile
\State \Return $t$
\EndFunction
\end{algorithmic}
\end{algorithm}
