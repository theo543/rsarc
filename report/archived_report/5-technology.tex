\chapter{Technology}

The project is implemented with Rust, and uses external libraries for OS interaction, multithreading, progress reporting, and \texttt{blake3} hashing.

\section{MultiThreaded Encoding and Decoding}

The same core code is used for encoding and repairing, as the same fundamental interpolation and evaluation process is used in both cases.

When generating parity data, symbols are read strictly from the data file and written strictly to the parity file.
When repairing data, in general, symbols are read from both files, and written to the damaged files, which could be either or both of the data and parity files.

The core code is given the indices of good and corrupt blocks in both files, the input and output x values to use for encoding,
handles of the input and output files, and memory maps of the same files.

In the case of encoding, it is told to use all blocks in the data file, and consider every parity block corrupt (as none exist yet).
For repair, the verification code is used to determine which blocks are corrupt using the hashes.

Reading is done using the \texttt{positioned-io} library, which allows convenient random access to files.
Attempting to use memory maps for reading seemed to cause the file to be read into memory and remain there, with old pages not being removed from memory.

For writing, the memory maps, created with the \texttt{memmap2} library, are used instead.

Communication between threads is primarily done using \texttt{crossbeam-channel}, a library which provides multi-producer, multi-consumer channels.

The multithreaded pipeline consists of a reader thread, an adapter thread, many processor threads, and a writer thread.

Five channels are used for the following purposes:
\begin{itemize}
    \item Sending filled input buffers from the reader to the adapter.
    \item Sending filled input buffers, with some additional data and reference counting added, from the adapter to the processors.
    \item Sending filled output buffers from the processors to the writer.
    \item Returning input buffers to the reader after they have been processed by the processors.
    \item Returning output buffers to the processors after their data has been written by the writer.
\end{itemize}

Since multiple codes are read into an input buffer at once, reference counting and read-write locks are used to manage the sharing of input buffers between multiple processor threads.
To return input buffers to the reader, an atomic integer is bundled with the buffer, and decremented by a processor when it finished interpolating a code from the buffer,
so that the last processor to work on a buffer knows to return it to the reader.

The adapter thread is responsible for sending many references to the same input buffer to the processors,
wrapped in a structure that includes information about which code from the buffer to process,
and a reference to the atomic integer used to count the number of tasks remaining for the buffer.

Unlike input buffers, output buffers contain a single code. The writer relies on the operating system memory mapping system to efficiently write the data to disk, coalescing writes when possible.
While it would be possible to gather output buffers into larger buffers and use \texttt{positioned-io} for writing as well as reading, the operating system appears to handle the write-only maps efficiently,
and using memory maps for writing was simpler to implement.

The amount of memory and number of threads used is automatically determined, using the libraries \texttt{num\_cpus} and \texttt{sysinfo} to query the OS for the available resources.

\section{User Interface}

The program has a basic CLI interface, implemented without any libraries.
It supports the following commands:
\begin{itemize}
    \item \texttt{encode} - Generates parity data.
    \item \texttt{verify} - Checks for corruption in the data and parity files. Code shared with the repair command for finding corrupt block locations.
    \item \texttt{repair} - Repairs corruption in the data or parity files, if there is enough redundancy.
    \item \texttt{reassemble} - Attempts to find misplaced blocks in the data and parity files, and copies them to new files in the correct locations.
    \item \texttt{test} - Runs an end-to-end test of the encoding, verification, and repair commands. (default)
\end{itemize}

Progress reporting is done using the terminal progress bar library \texttt{indicatif}.

\section{Testing}

The finite field and polynomial arithmetic code is testing with random data (generated using \texttt{fastrand}) using Rust's built-in testing framework.

The encoding, verification, and repair code is tested using the aforementioned end-to-end test, which randomly corrupts a test file and attempts to repair it.

Automated testing has not yet been implemented for the \texttt{reassemble} command. Manual testing was successful.
