\chapter{Reed-Solomon Codes and Finite Fields}

\section{Reed-Solomon Codes}

Reed-Solomon codes are a well-known class of error-correcting codes used in a wide range of applications, from data storage to radio communication.
They are based on polynomials over finite fields. \cite{theory-of-error-correcting-codes}

The code used for this project is an erasure code over the field \GF{64}, implemented using $O(n \log n)$ algorithms introduced in \cite{novel-poly}.

The basic working principle of this type of Reed-Solomon code is the interpretation of data as values of a polynomial evaluated at points ${\omega_0, \omega_1, \ldots, \omega_{k - 1}}$ in a finite field $\text{GF}(2^n)$.
The polynomial is interpolated and evaluated to obtain additional points, which are the parity information used to correct errors.
The combination of data and parity points is called a codeword.

As there is only one polynomial of degree $k - 1$ or smaller passing through $k$ points, any combination of at least $k$ of the original and redundant points uniquely determines the original polynomial.

Because each point must have a different x-coordinate, the field size limits how much data and redundancy a single codeword can contain. In \GF{64}, the limit is effectively infinite.

An erasure code is a type of error-correcting code which requires that the locations of corrupted data are known.
The code cannot be used to discover the locations of corrupted data by itself.
In this case, hashes stored in the metadata of the parity file are used to determine error locations.

Other Reed-Solomon codes do locate errors without requiring hashes, but they are not used in this project, as hashes are a simpler and more efficient solution.

One common code is $\text{RS}(255, 223)$, which is used in CDs and DVDs, and uses 8-bit symbols (in the field $\text{GF}(2^8)$).
The notation $\text{RS}(n, k)$ denotes a code with $n$ total symbols, with $k$ data symbols and $n - k$ parity symbols.
The code used in this project has no fixed $n$ or $k$, they are specified by the user.

The code used in this project is also systematic, meaning that the original data is included in the output.
Non-systematic codes do not include the original data, so the receiver must decode the received codeword to obtain the original data, even if no corruption occurred.

\section{Finite Fields}

Finite fields, also known as Galois fields, are mathematical structures which define addition, multiplication, subtraction, and division over a finite set of elements \cite{finite-fields-2nd-ed}
(as opposed to the better-known infinite fields, such as the rationals, reals, and complex numbers).

A field must satisfy the following properties:

\begin{itemize}[nosep]
    \item Associativity of addition and multiplication: $(a + b) + c = a + (b + c)$ and $(a \cdot b) \cdot c = a \cdot (b \cdot c)$
    \item Commutativity of addition and multiplication: $a + b = b + a$ and $a \cdot b = b \cdot a$
    \item Additive and multiplicative identity elements: $a + 0 = a$ and $a \cdot 1 = a$
    \item Additive inverses: for every $a$, there exists $-a$ such that $a + (-a) = 0$
    \item Multiplicative inverses: for every $a \neq 0$, there exists $a^{-1}$ such that $a \cdot a^{-1} = 1$
    \item Distributivity of multiplication over addition: $a \cdot (b + c) = a \cdot b + a \cdot c$
\end{itemize}

The theorems of polynomial mathematics used in Reed-Solomon codes only hold in a field, however standard computer arithmetic does not form a field.
Typical arithmetic supported natively by CPUs is fixed-size binary arithmetic with overflow, which is equivalent to arithmetic modulo a power of 2.
Modular arithmetic only forms a field with a prime modulus, so it cannot be used directly for Reed-Solomon codes.

For example, the operation $x \cdot 2$ is not invertible, as it is equivalent to a left shift, from which the most significant bit of $x$ cannot be recovered.

Fortunately, it is possible to construct a field based on fixed-size integers, such as 64-bit integers.

In a finite field $\text{GF}(p^m)$, where $p$ is a prime number and $m$ is a positive integer, the elements are polynomials of degree $m - 1$, with coefficients in $\text{GF}(p)$.
For the \GF{n} case, an element in the field is a polynomial with $n$ coefficients, where each coefficient is a bit (i.e. a value in $\text{GF}(2)$ = $\{0, 1\}$).
For the purposes of this project, $n$ is always 64, so the field is \GF{64}.

The notation $\omega_i$ is used to denote the integer $i$ converted to an element of the field \GF{64} by interpreting its bits as a polynomial, which is a no-op in code, as elements of \GF{64} are stored as 64-bit integers.

It is important to note that these polynomials are not the same as the ones used in Reed-Solomon codes to represent data and parity information.
Elements of \GF{n} are simply $n$ bit integers with more complex arithmetic. They are polynomials over $\text{GF}(2)$, with $n$ coefficients.
Reed-Solomon polynomials can be arbitrarily long. They are polynomials over \GF{64}, with an arbitrary number of coefficients, and each coefficient is itself a polynomial over \GF{2} with $64$ coefficients.

Finite field addition is defined as polynomial addition.
In a general field $\text{GF}(p^m)$, this would be implemented as pairwise addition of the coefficients of two polynomials, modulo $p$.

In binary finite fields, addition is equivalent to XOR, as the coefficients are bits. Therefore, $x + x = 0$, and $x = -x$ (the field has characteristic 2).

Multiplication is defined as polynomial multiplication, followed by reduction modulo an irreducible polynomial of degree 64 (with $65$ coefficients, where the highest coefficient is $1$).

The irreducible polynomial used for this project is $x^{64} + x^4 + x^3 + x + 1$ \cite{low-weight-polynomials}.
The choice of irreducible polynomial does not affect correctness, and the fields obtained from different choices are isomorphic.

\section{Efficient Finite Field Arithmetic}

The efficiency of finite field arithmetic is crucial for the performance of the encoding and decoding algorithms.

Addition is simple and fast, as it is equivalent to XOR.
Multiplication, however, is slower and more complicated than integer multiplication, and division even more so.

\subsection{Russian Peasant Algorithm}

The Russian peasant algorithm multiplies two values in \GF{64} without requiring 128-bit integers.
It incrementally performs the multiplication by adding intermediate values into an accumulator, and slowly shifting the values to be multiplied and applying polynomial reduction.

This algorithm is fairly simple and easy to implement, however multiplication can be done far more efficiently on modern CPUs using carry-less multiplication.
Still, this algorithm is necessary as a fallback for older CPUs.

The state of the algorithm consists of the two values to be multiplied $a$ and $b$, and an accumulator.

At each iteration, if the low bit of $b$ is set, the accumulator is XORed with $a$.
Then, $a$ is shifted left, and $b$ is shifted right.

This is justified because, at each step, we multiply the lowest coefficient of $b$ with $a$, and add the result (either $0$ or $a$) to the accumulator.
Then, moving on to the next coefficient of $b$, we divide $b$ by $x$ and multiply $a$ by $x$.

If the high bit of $a$ was set before shifting, $a$ is XORed with the irreducible polynomial, excluding the highest coefficient which would not fit in 64 bits.
Conceptually, $a$ now has a 65th bit (a coefficient $x^{64}$), which is removed by subtracting the irreducible polynomial.

\begin{algorithm}
\caption{Russian Peasant Multiplication}
\begin{algorithmic}
\Function{Multiply}{$a, b$}
\State $acc \gets 0$
\For{$\text{i} \gets 1 \text{ to } 64$}
    \If{$b \bitand 1$}
        \State $acc \gets acc \oplus a$
    \EndIf
    \State $\text{carry} \gets a \bitand (1 \ll 63)$
    \State $a \gets a \ll 1$
    \State $b \gets b \gg 1$
    \If{$\text{carry}$}
        \State $a \gets a \oplus \text{POLYNOMIAL}$ 
    \EndIf
\EndFor
\State \Return $acc$
\EndFunction
\end{algorithmic}
\end{algorithm}

\subsection{Carry-less Multiplication}

\GF{64} multiplication can be performed using only three 128-bit carry-less multiplication operations.
Modern CPUs have support for this operation, as it is useful for cryptographic algorithms, computing checksums, and other applications. \cite{intel-clmul}

The terms "upper half" and "lower half" will be used to refer to the most significant 64 bits and least significant 64 bits of a 128-bit integer, respectively.

By multiplying $a$ and $b$ using carry-less multiplication, we obtain a 128-bit result.
We must reduce the upper half to a 64-bit result, which can then be XORed with the lower half to obtain the final result.

This can be done by multiplying the upper half of the result by the irreducible polynomial.
Then, the lower half of the result is the product reduced modulo the irreducible polynomial.

To understand why this works, consider the process of reduction.
The irreducible polynomial is aligned with each set bit in the upper half of the result, and XORed with the result.
This is effectively what carry-less multiplication does.

There is a complication, however.
A third multiplication is required to ensure full reduction, as the highest bits of the upper half can affect the lowest bits of the upper half.

For fields where $x^{n - 1} + 1$ is irreducible, two multiplications would suffice, but this is not the case for \GF{64}.

For example, consider $x^{127} + x^{67} + x^{66} + x^{64}$.
After aligning the irreducible polynomial with the highest bit and XORing, all bits in the upper half are zero.
At this point, the reduction is complete and should stop.
However, this is not how carry-less multiplication works.
The irreducible polynomial will also be aligned with the other three bits, and the lower half is XORed with some unnecessary values.

The upper half of the reduced result if and where this happened. A third multiplication will correct this.
The unnecessary XORs are undone by XORing with the lower half of the third multiplication.

The justification for the algorithm may seem somewhat complex, but the algorithm itself is very short, simple, and efficient.

The functions $\text{upper}(x)$ and $\text{lower}(x)$ return the upper and lower 64 bits of the 128-bit integer $x$, respectively.

\begin{algorithm}
\caption{Carry-less Multiplication}
\begin{algorithmic}
\Function{Multiply}{$a, b$}
\State $\text{result} \gets \text{CLMUL}(a, b)$
\State $\text{result\_partially\_reduced} \gets \text{CLMUL}(\text{upper}(\text{result}), \text{POLYNOMIAL})$
\State $\text{result\_fully\_reduced} \gets \text{CLMUL}(\text{upper}(\text{result\_partially\_reduced}), \text{POLYNOMIAL})$
\State \Return $\text{lower}(\text{result}) \oplus \text{lower}(\text{result\_partially\_reduced}) \oplus \text{lower}(\text{result\_fully\_reduced})$
\EndFunction
\end{algorithmic}
\end{algorithm}

\subsection{Extended Euclidean Algorithm}

A simple way to compute the multiplicative inverse is to raise the element to the power of $2^{64} - 2$ using exponentiation by squaring.
This was used in the early stages of this project, however it is not the most efficient method.
It is still useful for unit testing, by checking that the more complicated algorithm agrees with the simpler exponentiation method.

The polynomial extended Euclidean algorithm, given polynomials $a$ and $b$, computes $s$ and $t$ such that $a \cdot s + b \cdot t = \text{gcd}(a, b)$.
When $b$ is set to the irreducible polynomial, $t$ is the multiplicative inverse of $a$. $s$ does not need to be computed.

The algorithm uses repeated Euclidean division.
Because the irreducible polynomial is of degree 64, the first Euclidean division iteration, in the first iteration of the Euclidean algorithm, is a special case.
As a 65-bit polynomial cannot fit in the 64-bit variable $b$, the first iteration is done manually, outside the loop.

In the following pseudocode, $\text{leading\_zeros}(x)$ returns the number of leading zero bits in $x$.
Modern CPUs have a dedicated instruction for counting leading zeros.

\begin{algorithm}
\caption{Extended Euclidean Algorithm}
\begin{algorithmic}
\Function{ExtendedEuclidean}{$a$}

\State $\text{assert}(a \neq 0)$

\State \algorithmicif\ $a = 1$ \algorithmicthen\ \Return $1$ \algorithmicend \algorithmicif

\State $t \gets 0$
\State $\text{new\_t} \gets 1$
\State $r \gets \text{POLYNOMIAL}$
\State $\text{new\_r} \gets a$

\State $r \gets r \oplus (\text{new\_r} \ll (\text{leading\_zeros}(\text{new\_r}) + 1))$
\State $\text{quotient} \gets 1 \ll (\text{leading\_zeros}(\text{new\_r}) + 1)$

\While{$\text{new\_r} \neq 0$}
    \While{$\text{leading\_zeros}(\text{new\_r}) >= \text{leading\_zeros}(r)$}
        \State $\text{degree\_diff} \gets \text{leading\_zeros}(\text{new\_r}) - \text{leading\_zeros}(r)$
        \State $\text{r} \gets r \oplus (\text{new\_r} \ll \text{degree\_diff})$
        \State $\text{quotient} \gets \text{quotient} | (1 \ll \text{degree\_diff})$
    \EndWhile
    \State $(r, \text{new\_r}) \gets (\text{new\_r}, r)$
    \State $(t, \text{new\_t}) \gets (\text{new\_t}, t \oplus \text{gf64\_multiply}(\text{quotient}, \text{new\_t}))$
    \State $quotient \gets 0$
\EndWhile
\State \Return $t$
\EndFunction
\end{algorithmic}
\end{algorithm}
