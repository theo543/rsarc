\chapter{Polynomial Interpolation and Evaluation}

\section{Horner's Rule}

Horner's rule is a method for evaluating polynomials in $\bigO(n)$ time.

A polynomial $a_0 + a_1 x + \ldots + a_{n-1} x^{n-1}$ can be written as $a_0 + x(a_1 + x(a_2 + \ldots + x(a_{n-1}) \ldots))$.

It can then be evaluated from the inside out, starting with the highest degree coefficient $a_{n-1}$,
and repeatedly multiplying by $x$ and adding the next coefficient.

\section{Newton Interpolation}

Polynomial interpolation is used to find a polynomial of degree at most $k - 1$ that passes through $k$ points, where the x coordinates of the points are distinct.
It is known that this polynomial is unique.

Newton interpolation is the polynomial interpolation algorithm used in this project to interpolate data in $\bigO(n^2)$ time and $\bigO(n)$ space, where $n$ is the number of points.

The Newton interpolation polynomial is a linear combination of basis polynomials.

\begin{equation}
    N(x) = \sum_{k=0}^{n} \;\; [y_0, \ldots, y_k] \cdot \prod_{i=0}^{k-1} (x - x_i)
\end{equation}

The notation $[y_k, \ldots, y_{k+j}]$ denotes the divided difference, which is recursively defined as follows:

\begin{equation}
    [y_k] = y_k
\end{equation}

\begin{equation}
    [y_k, \ldots, y_{k+j}] = \frac{[y_{k+1}, \ldots, y_{k+j}] - [y_k, \ldots, y_{k+j-1}]}{x_{k+j} - x_k}
\end{equation}

Naive computation of divided differences requires $\bigO(n^2)$ time and space, but the space requirement can be reduced to $\bigO(n)$ by reusing values from previous iterations.

At the initial iteration, the basis polynomial is initialized to $n_0(x) = 1$ and the list of divided differences is initialized to $[y_0]$.

At any iteration $k + 1$, divided differences $[y_0, \ldots, y_k], [y_1, \ldots, y_k], \ldots, [y_{k-1}, y_k], [y_k]$ are known from previous iterations.

The value $[y_{k+1}]$ is added to the list of divided differences, and the rest of the values are updated to
$[y_0, \ldots, y_{k+1}], [y_1, \ldots, y_{k+1}], \ldots, [y_{k-1}, y_{k+1}], [y_k, y_{k+1}]$, from right to left.
At every step of the update, the next divided difference is computed using the current value and the value to its right (which was updated in the previous step).

The basis polynomial is $n_{k+1}$ is computed by multiplying the previous basis polynomial by $x - x_k$.

The interpolation polynomial is updated by adding the basis polynomial multiplied by the divided difference to it.

At each iteration, we perform multiplication by $x - x_k$, update the divided differences, and update the interpolation polynomial.
Since these steps require $\bigO(k)$ time, and we perform $n$ iterations, the total time complexity is $\bigO(n^2)$.

No general polynomial multiplication is required, nor is any polynomial division (only \GF{64} division in the divided differences).
